%!TEX encoding = UTF-8 Unicode
%-------------------------------------------------------------------------------
%	SECTION TITLE
%-------------------------------------------------------------------------------
\cvsection{Work Experience}


%-------------------------------------------------------------------------------
%	CONTENT
%-------------------------------------------------------------------------------
\begin{cventries}
%---------------------------------------------------------
  \cventry
  {Backend Engineer} % Job title
  {AO Labs} % Organization
  {Seoul, S.Korea} % Location
  {May. 2024 - Present} % Date(s)
  {
      \begin{cvitems} % Description(s) of tasks/responsibilities
        \item[] {\textbf{Market service} \hspace{1cm} \textit{Go, Redis (Pubsub \& Stream), TimescaleDB (PostreSQL)}}
        \item {토큰의 가격을 비롯한 시장데이터를 유동적으로 실시간 구독하고 다루는 서비스 설계 및 구현.}
        \item[] 
        \item[] {\textbf{Backend of a Blockchain-based Service} \hspace{1cm} \textit{Go, gRPC, GCP, Cloud Run, GCP Pubsub, MySQL}}
        \item {Blockchain 데이터 구독 및 인덱싱.}
        \item {low-code 플랫폼(Retool)을 활용한 Admin tool 개발.}
      \end{cvitems}
  }
  
%---------------------------------------------------------
  \cventry
    {Backend Engineer, Team Lead} % Job title
    {Mathpresso, Inc.} % Organization
    {Seoul, S.Korea} % Location
    {Jan. 2022 - May. 2024} % Date(s)
    {
      \begin{cvsubentries}
        \cvsubentry{}{\small \underline{Lab Development Team Lead}}{Oct. 2022 - Mar. 2023}{}
        \begin{cvitems}
          \item[] {5명 정도의 백엔드 엔지니어와 함께 QADNA 서비스의 문제 검색엔진과 데이터 수집부터 인덱싱 담당.}
          \item {팀의 성과를 위해 개개인의 강점을 최대화하고자 노력.}
          \begin{itemize}
            \item {주기적인 1:1 미팅에서의 관계 형성을 바탕으로 강점 파악 및 커리어 목표 수립.}
          \end{itemize}
          \item {팀 생산성을 올리기 위해 빠르고 원활한 커뮤니케이션 환경 조성}
          \begin{itemize}
            \item {회의 효율화와 긍정적인 분위기 형성}
          \end{itemize}
          \item[]
        \end{cvitems}
        \cvsubentry{}{\small \underline{Content Platform TF Lead}}{Mar. 2023 - May. 2024}{}
        \begin{cvitems}
          \item[] {대량의 고품질 문제/해설 데이터를 효율적으로 생성 및 관리가 목표. \\ 엔지니어, 도메인 전문가, 서비스 운영자가 유기적으로 구성된 7명정도 규모의 팀 운영.}
          \item {다양한 이해관계자와 협업하며 전략적 비즈니스 의사결정에 기여.}
          \item {여러 미션에 따라 팀 자원을 활용하면서 팀원들이 공동의 목표를 갖도록 지원.}
          \begin{itemize}
            \item {정량적인 핵심 지표를 설정.}
            \item {목표달성에 따른 기대효과 제시}
          \end{itemize}
          \item[] 
        \end{cvitems}
      \end{cvsubentries}
    }

%---------------------------------------------------------
\cventry
{} % Job title
{} % Organization
{} % Location
{} % Date(s)
{
  \begin{cvsubentries}
    \cvsubentry{}{\small \underline{Backend Engineer}}{Jan. 2022 - May. 2024}{}
    \begin{cvitems} % Description(s) of tasks/responsibilities
      \item[] {\textbf{Knowledge Graph} \hspace{1cm} \textit{Go, Neo4j (graphDB)}}
      \item[] {수학교육에서의 주요 모델간의 관계를 유연하게 관리하는 프로젝트. (문제유형과 개념, 선행 개념과 후행 개념 등등)}
      \item {기획부터 참여하여 개념단위부터 커리큘럼까지 시스템화. 문제 라벨링을 비롯하여 개념기반 데이터에 활용.}
      \item[] 
      \item[] {\textbf{Data Pipeline} \hspace{1cm} \textit{Go, MySQL, GCP, AppEngine}}
      \item {검색엔진, AI 학습, 컨텐츠 서비스를 위한 데이터 수집, 가공 및 파이프라인 개발.}
      \item[] 
      \item[] {\textbf{Problem/Solution Resource Service} \hspace{1cm} \textit{Go, MySQL, Neo4j}}
      \item {문제/해설 데이터 관리 및 고도화.}
      \item[] 
      \item[] {\textbf{Image Search} \hspace{1cm} \textit{Go, Milvus (Vector DB)}}
      \item {사내의 Image Embedding 모델과 벡터 DB를 활용하여 이미지 검색 서비스 구현.}
      \item[]
    \end{cvitems}
  \end{cvsubentries}
}
%---------------------------------------------------------
  \cventry
    {Backend Engineer} % Job title
    {Kasa Korea Co., Ltd.} % Organization
    {Seoul, S.Korea} % Location
    {Jul. 2019 - Sep. 2021} % Date(s)
    {
      \begin{cvitems} % Description(s) of tasks/responsibilities
        \item[] {\textbf{Kasa Service Web Server.} \hspace{1cm} \textit{Django, DRF, MySQL, redis}}
        \item {서비스 초기 기획부터 운영까지 전반적인 개발 경험.}
        \item {인증, 공모, 원장(Ledger) 서비스를 구현하고, 주요 모델 설계.}
        \item[]
      \end{cvitems}
    }

%---------------------------------------------------------
  \cventry
    {Backend Engineer} % Job title
    {Omnious. Co., Ltd.} % Organization
    {Seoul, S.Korea} % Location
    {Nov. 2017 - Jul. 2019} % Date(s)
    {
      \begin{cvitems} % Description(s) of tasks/responsibilities
        \item[] {\textbf{Labeling Tool}}
        \item {AI 모델 학습용 패션 데이터를 수집하는 사내 웹 애플리케이션의 백엔드 유지 보수 (50명 이상의 활성 사용자).}
        \begin{itemize}
          \item[] \textit{Docker, CircleCI, Rancher, AWS(EC2, S3, RDS)}
        \end{itemize}
        \item {RESTful HTTP API 설계 및 서버 개발, API 문서화로 효율적인 커뮤니케이션 지원.}
        \begin{itemize}
          \item[] \textit{flask, NGINX, uWSGI, Swagger}
        \end{itemize}
        \item {도메인 모델의 관계 설정 및 데이터베이스 스키마 설계.}
        \begin{itemize}
          \item[] \textit{MySQL}
        \end{itemize}
        \item {HTTP 접근 로깅 시스템 및 데이터 저장 파이프라인 구축.}
        \begin{itemize}
          \item[] \textit{fluntd, fluent-bit}
        \end{itemize}
        \item {유닛 테스트 및 코드 스타일 가이드 도입으로 코드 품질 향상.}
        \begin{itemize}
          \item[] \textit{flake8, pytest}
        \end{itemize}
        \item[]
        \item[] {\textbf{Data Extraction \& Transformation} \hspace{1cm} \textit{pandas, Celery}}
        \item {수집된 데이터를 AI 모델 학습용으로 전처리 및 최적화.}
        \item {데이터 추출부터 처리까지의 프로세스 자동화 및 CLI 도구 개발.}
        \item[]
        \item[] {\textbf{Crawler} \hspace{1cm} \textit{Scrapy, ScrapingHub, Zappa, AWS(SNS, SQS, Labmda)}}
        \item {20개 이상의 패션 이미지 크롤러 개발 및 스케줄링 관리 도구 개발.} 
        \item {서버리스 파이프라인을 구축하여 크롤링된 데이터를 효율적으로 저장.}
        \item[]
      \end{cvitems}
    }

%---------------------------------------------------------
  % \cventry
  %   {Software Developer Intern} % Job title
  %   {LayGen Inc.} % Organization
  %   {Gyeonggi-do, S.Korea} % Location
  %   {Aug. 2017 - Sep. 2017} % Date(s)
  %   {
  %     \begin{cvitems} % Description(s) of tasks/responsibilities
  %       \item {Participated to manufactoring operating module for a prototype of 3D printer and developed software control the module.}
  %       \item {Developed GUI so that hardware engineer can test controling each part of 3D printer.}
  %       \item[--] \textit{Python, PyQT, Rasberry Pi, Arduino}
  %     \end{cvitems}
  %   }

%---------------------------------------------------------
  \cventry
    {Software Developer Intern} % Job title
    {LINE Corp.} % Organization
    {Gyeonggi-do, S.Korea, S.Korea} % Location
    {Apr. 2017 - May. 2017} % Date(s)
    {
      \begin{cvitems} % Description(s) of tasks/responsibilities
        \item[] {\textbf{Targeted Video} \hspace{1cm} \textit{Swift, OpenCV}}
        \item[] {영상에서 원하는 Object를 분리 및 변조하는 iOS 앱 개발.}
        \item {Contour extraction, Image segmentation and Stabilization 등의 기술 사용.}
        \item {선택한 객체를 자르거나 붙이는 비디오 편집 기능 구현.}
      \end{cvitems}
    }

%---------------------------------------------------------
  % \cventry
  %   {Software Developer Intern} % Job title
  %   {UXEnterprise} % Organization
  %   {Seoul, S.Korea} % Location
  %   {Aug. 2016 - Sep. 2016} % Date(s)
  %   {
  %     \begin{cvitems} % Description(s) of tasks/responsibilities
  %       \item {Prototyped Android application viewing Female biorhythm.}
  %       \item {Implemented connection with a measuring device using Bluetooth.}
  %       \item[--] \textit{Java, Android}
  %     \end{cvitems}
  %   }

%---------------------------------------------------------
\end{cventries}
